\resumo

A Pintura Hidrográfica, \textit{Water Transfer Printing} ou \textit{Hydrographics} é uma técnica viável para colorir objetos criados em impressoras 3D. Porém, neste tipo de pintura há uma complexa interação entre um filme e um objeto impresso em 3D, na qual a imagem contida no filme deve aderir ao objeto. Este processo está sujeito a incertezas sobre como o filme se deformará para aderir ao objeto impresso em 3D e transferir a imagem para a superfície do objeto, conferindo-lhe um acabamento resultado de um mapeamento de uma imagem 2D em um modelo real 3D.

Diante destas dificuldades, será apresentada uma proposta de simulação computacional 3D para a técnica de pintura hidrográfica. Com base na interação física entre os corpos envolvidos neste fenômeno, será proposta uma simulação computacional 3D que mostre o que ocorre durante a pintura hidrográfica. Desta forma, será dado ao usuário um suporte ao usuário para antever o processo da pintura em formas mais complexas, que necessitem de alinhamentos com a imagem transferida ao objeto. Além disso, o trabalho também busca simular o fenômeno físico da pintura hidrográfica em tempo real, fazendo uma avaliação da utilização dos recursos computacionais da \acs{GPU} (\acl{GPU}) empregados nesta simulação.
 
% Palavras-chave do resumo em Português
\begin{keywords}
Impressão 3D, pintura hidrográfica, simulação baseada em física, renderização em tempo real, mapeamento de texturas 
\end{keywords}
