\xchapter{Conclusões}{Síntese deste texto, com algumas considerações preliminares}

\acresetall 

\section{Conclusões preliminares}

Os resultados iniciais da simulação que foi implementada são bastante promissores. Ainda não houve uma avaliação entre a simulação e o experimento físico. Contudo, acredita-se que com alguns ajustes importantes, será possível desenvolver uma simulação próxima do fenômeno real.

Será necessário aprimorar a técnica, para que seja possível simular a pintura em qualquer objeto e não somente uma esfera. A ideia será implementar um sistema de detecção de colisão entre os triângulos de uma malha qualquer (representando o objeto) com os pontos da malha plana (que representa o filme). Será preciso projetar um sistema que seja eficiente em termos de desempenho.

O controle da cor também não foi implementado. Durante o experimento físico, percebe-se que algumas regiões do filme sofrem um maior esticamento do que outras, o que no leva a um clareamento da cor. É desejável que após a simulação, seja impressa no filme uma imagem que possa amenizar estas distorções, gerando uma compensação inversa (aumentando a pigmentação nas regiões em que se preveja um maior esticamento). Para isto, será necessário pesquisar métodos de simular a variação da espessura do filme e propor uma função que torne a cor mais escura na medida em que o filme esteja em uma região de maior esticamento.

A simulação permite que o usuário veja como ficaria um objeto se submetido a pintura hidrográfica. Porém, ainda não permite preparar o filme com a imagem distorcida, que será aplicada ao objeto. Um passo importante seria gerar o mapeamento dos pontos do objeto no filme (chamado usualmente de função inversa de mapeamento). Assim, tendo um objeto conhecido, seria possível mapear no filme os pontos-chave do objeto em que se deseje imprimir uma textura. Para isto, a proposta seria determinar uma função de aproximação, que leve pontos médios de correspondência entre os vértices da malha deformada e os vértices do objeto.

A pesquisa tem o potencial de proporcionar um desenvolvimento nas técnicas de dar acabamento a objetos criados em impressoras 3D. O potencial da impressão 3D colorida é bastante amplo, tornando-a aplicável em inúmeras situações em que se necessite automatizar a reprodução precisa de formas complexas em 3D a um baixo custo. Protótipos criados em impressão 3D têm sido constantemente utilizados como prova de conceito, viabilizando projetos. Talvez por, por este motivo, esta tecnologia foi ganhando outras aplicações: na área médica para geração de próteses, na indústria de entretenimento gerando miniaturas de personagens de filmes e jogos, na confecção de vestimentas e calçados.

No contexto da preservação de artefatos culturais, a impressão 3D colorida vem crescendo, sobretudo no auxílio a preservação de ativos históricos. Atualmente, é possível capturar um objeto histórico em três dimensões e representá-lo em um modelo 3D. Com este objeto capturado e representado virtualmente, é possível recriá-lo em três dimensões. No caso de uma peça que esteja incompleta pela ação do tempo, é possível modelar partes faltantes da peça e recompô-la com o auxílio da tecnologia.

Nos dias de hoje, museus que possuem um acervo digitalizado em 3D podem facilmente reproduzir peças deste acervo e colocá-las em exposição, mantendo a peça original resguardada em um outro ambiente, evitando a degradação pela ação do tempo. Peças originais e portanto, raras, devem estar protegidas de qualquer intempérie. Artefatos importantes devem estar em segurança, evitando um dano por algum tipo de acidente ou catástrofe natural. De forma complementar, um usuário pode tocar uma peça impressa em 3D parecida com a original, tendo contato com uma experiência sensorial muito mais completa. Museus históricos podem oferecer como souvenir a réplica da miniatura de um artefato representativo, difundindo ainda mais a experiência tida em uma visita.
