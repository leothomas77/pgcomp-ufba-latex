\appendix

\xchapter{Detalhamento do Experimento Preliminar}{Aqui serão detalhados os materiais e os passos para execução do experimento preliminar com a pintura hidrográfica}

\acresetall 

\section{Material utilizado}
\label{apendice1:material}

Foi adquirido um material básico para realização do experimento, conforme a lista que segue:

\begin{enumerate}[label=\alph*)]
\item Container plástico de 20 litros retangular
\item Máscara de proteção
\item Luvas de proteção
%\item Aquecedor para aquário
\item 200 ml de ativador tipo A
\item Borrifador
%\item 400 ml de ativador tipo B
\item 1 rolo de fita crepe
\item 3 folhas de película hidrográfica com fundo transparente e estampa prateada de carbono (dimensões 100 x 50 cm)
%\item 20 folhas de película hidrográfica em banco própria para impressão (dimensão A4)
%\item Impressora com cartucho de tinta pigmentada (ou serviço de impressão)
\end{enumerate}

\section{Etapas do experimento}
\label{apendice1:etapas}

\begin{enumerate}
\item Encher o container plástico com água morna entre 27\textdegree{C} a 30\textdegree{C}. Esperar o movimento da água cessar. 
\item Obter um pedaço de película com dimensões compatíveis com container. Deve-se colocar uma moldura com fita crepe nas bordas da película, afim de conter a sua expansão quando ativada. 
\item Dispor a película suavemente no tanque. Sem deixar formar bolhas por debaixo da película. Uma técnica prática consiste em segurar a película pelas extremidades de sua diagonal e ir dispondo-a suavemente.
\item Observar o movimento inicial da película: inicialmente ela ficará enrugada, depois ficará mais plana.
\item Borrifar aos poucos uma pequena quantidade de ativador tipo A em toda extensão da película. No experimento realizado, foram dadas 6 borrifadas. Observar a reação da película. A película deve se dissolver gradativamente, até adquirir um aspecto brilhante na superfície da água. Neste momento, estará pronta para a imersão do objeto.
\item Submergir lentamente o objeto de forma oblíqua em relação a linha d'água. Recomenda-se que o objeto esteja a um ângulo de aproximadamente 30$^{\circ}$ em relação à linha d'água.
\item Ao finalizar a imersão, agitar o objeto para romper a película residual (que não será útil na pintura).
\item Aguardar de quatro a seis horas a secagem completa da peça.
\end{enumerate}

Como o ativador possui um odor forte semelhante a um solvente, fez-se necessário o uso de máscara de proteção.
