\xchapter{Introdução}{Este é o primeiro capítulo, onde será dada uma contextualização sobre o trabalho, a apresentação do problema, hipótese de pesquisa e propostas de solução.}

% É recomendável utilizar `\acresetall' no início de cada capítulo para reiníciar o contator de referências às siglas.
\acresetall 

\section{Introdução}

\subsection{Contexto}

A impressão 3D é um processo em que se utiliza recursos computacionais e eletro-mecânicos para a fabricação aditiva de peças tridimensionais a partir de modelos digitais.
Hoje, a produção de peças em três dimensões com a impressão 3D é uma área em constante expansão, suprindo diretamente a demanda da indústria para a prototipação rápida de produtos nas engenharias e na arquitetura. A técnica evoluiu bastante nos últimos anos, de modo que passou a ser disponibilizada a um custo cada vez menor. Se antes era uma tecnologia restrita aos grandes centros tecnológicos de pesquisa, agora passou a estar também disponível em ambientes de menor porte ou mesmo no ambiente doméstico, levando à descoberta de novas aplicações a cada dia.

O potencial da impressão 3D é bastante amplo, tornando-a aplicável em inúmeras situações em que se necessite automatizar a reprodução precisa de formas complexas em 3D a um baixo custo. Protótipos criados em impressão 3D têm sido constantemente utilizados como prova de conceito, viabilizando projetos. Talvez por, por este motivo, esta tecnologia foi ganhando outras aplicações: na área médica para geração de próteses, na indústria de entretenimento gerando miniaturas de personagens de filmes e jogos, na confecção de vestimentas e calçados.

No contexto da preservação de artefatos culturais, a impressão 3D vem crescendo, sobretudo no auxílio a preservação de ativos históricos. Atualmente, é possível capturar um objeto histórico em três dimensões e representá-lo em um software de modelagem em 3D. Com este objeto capturado e representado virtualmente, é possível recriá-lo em três dimensões. No caso de uma peça que esteja incompleta pela ação do tempo, é possível modelar partes faltantes da peça e recompô-la com o auxílio da tecnologia.

Nos dias de hoje, museus que possuem um acervo digitalizado em 3D podem facilmente reproduzir peças deste acervo e colocá-las em exposição, mantendo a peça original resguardada em um outro ambiente, evitando a degradação pela ação do tempo. Peças originais e portanto, raras, devem estar protegidas de qualquer intempérie. Artefatos importantes devem estar em segurança, evitando um dano por algum tipo de acidente ou catástrofe natural. De forma complementar, um usuário pode tocar uma peça impressa em 3D idêntica a original, tendo contato com uma experiência sensorial muito mais completa. Museus históricos podem oferecer como souvenir a réplica da miniatura de um artefato representativo, difundindo ainda mais a experiência tida em uma visita.

\subsection{Justificativa}

Apesar de toda a evolução tecnológica já alcançada para a fabricação de formas complexas, reproduzir fielmente as cores de um objeto impresso em 3D ainda é um grande desafio. A maioria das impressoras em 3D disponíveis no mercado geram objetos com somente uma cor. Há alguns dispositivos que geram a forma e incorporam a cor durante o processo, tornando a  geração da forma indissociável da pigimentação. Isto torna a opção indesejável, pois restringe a opção a poucos modelos, tirando-se a liberdade de se escolher o equipamento com o menor custo possível.

Dentre as opções para se colorir o material existe a pintura tradicional com aerógrafo, rolo ou pincel, que requerem um domínio de técnicas, e um bom conhecimento dos materiais envolvidos (tipo de tinta adequado para o tipo de substrato, diluentes, preparação da superfície, aplicação de camadas, tempo de cura e acabamento). A pintura tradicional requer um espaço físico adequado. O ambiente deve ser controlado, com temperatura e umidade apropriados para que todas etapas sejam bem sucedidas. Parte do material usado na pintura é desperdiçado, seja por aplicação da tinta em partes extras (provocando sujeira), por conta de mau dimensionamento do material, ou do desperdício por conta do usuário. É um processo que depende bastante de se ter pessoas qualificadas à disposição, aumentando-se o custo de mão de obra. Técnicas tradicionais de pintura geram um resultado diferente para cada peça, dificultando a previsibilidade do trabalho. A natureza da técnica faz com que cada peça pintada à mão resulte em um trabalho artístico único. 

A Pintura Hidrográfica é conhecida também pelos nomes Water Transfer Printing - WTP, hydrographic printing ou simplesmente hydrographics. Trata-se de uma técnica bastante difundida para pintura de superfícies em 3D. O processo da Pintura Hidrográfica tem origens pouco esclarecidas. No entanto, registrado sob patente em 1982, por Motoyasu Nakanishi. É descrita da seguinte maneira:

“Um mecanismo de pintura provido por uma estrutura que fornece um filme com um padrão a ser transcrito em uma banheira, na qual o filme permanece flutuando em um líquido. Uma estrutura lentamente submerge o objeto a ser pintado no líquido, de cima para baixo, fazendo com que o líquido contendo o padrão a ser transcrito flua pelo objeto de baixo para cima.” 
\cite{Nakanishi1984}, tradução nossa

Tornou-se uma técnica bastante conhecida na decoração de peças automotivas, customizações de itens de uso pessoal ou de itens de acabamento. Pode ser aplicado em diversos substratos como plástico, metal, fibra de vidro e cerâmica. Em princípio, qualquer material que pode ser submergido em um líquido, pode ser utilizado como substrato para a Pintura Hidrográfica.

O processo de pintura possui alguns passos bem definidos que se forem rigorosamente seguidos, produzem um acabamento de qualidade e fiel ao esperado. Em contraste com a pintura tradicional, a pintura hidrográfica é uma técnica mais limpa e que exige um ambiente mais simples de se manter.

devem ser seguidos, por recomendação do fabricante da película. Estes passos incluem uma preparação da superfície, e a aquisição de uma película com um padrão a ser impresso.
Com a superfície preparada, um filme polivinílico contendo a imagem a ser transferida é cuidadosamente disposto na superfície de um tanque contendo água. A este filme então é aplicado um ativador que o dissolve. Em seguida, o objeto é submergido. A superfície da água contém o pigmento da imagem a ser transferida. Este pigmento vai aderindo gradativamente ao longo da superfície, abraçando a forma submergida. Algum resíduo da película é descartado. O objeto agora está com a imagem aderida a sua superfície, que será seca.

Dentre as características desta técnica, está a possibilidade de dar acabamento para os mais variados tipos de materiais como fibra de vidro, madeira, metal, cerâmica, plástico. O objeto trabalhado portanto, pode ter sido gerado em qualquer equipamento.

A técnica possibilita transferir um acabamento para peças com formas complexas. O tipo de acabamento, pode ser igualmente complexo, podendo ser desde uma textura de cor plana até padronagens complexas, como texturas que imitam madeira, rocha, camuflagem, pele, couro e uma grande variedade de estampas que imitam texturas do mundo real.
A vantagem de se aplicar uma pintura hidrográfica em uma peça é que com esta técnica, é possível transferir para um objeto em 3D uma textura do mundo real sem a necessidade de dominar a técnica de pintura tradicional (que envolve aquisição de tinta, solvente, pincéis rolos ou spray) e sem os problemas decorrentes da pintura (a necessidade de preparação pode ser bem menor).

A pintura tradicional necessita da aquisição de um conjunto de ferramentas (fundo preparador, solventes, camadas de tinta, espaço para secagem, isolamento das partes a serem pintadas) que pode se tornar mais custoso do que o processo de hidrografia completo, que pode ser mais imediato, menos artístico e mais e industrial (embora traga um resultado considerado até artístico em nível de detalhes).
Além disso o processo de pintura pode ser diferenciado de acordo com o tipo de superfície, requerendo uma maior habilidade técnica do executor, seja no conhecimento da tinta apropriada, seja na própria aplicação do pigmento.
A reprodução de padrões mais complexos em superfícies 3D de forma consistente se dá de maneira bastante simples com a pintura hidrográfica, ao contrário de outras técnicas como spray,


No campo de estudo da computação visual e da computação gráfica, a reconstrução 3D é o campo de estudo que se dedica na aquisição de objetos em 3D, podendo capturar todos os detalhes de um objeto real para um modelo virtual em 3D. A reconstrução 3D estuda técnicas de captura e reprodução tanto da superfície quanto da aparência destes objetos (cor difusa, textura, e aspectos de reflectância)

Recentemente, junto com as técnicas de modelagem em 3D, tornaram-se acessíveis para o público técnicas que permitem a fabricação de modelos através de impressão em 3D. Deste modo, o pipeline de reconstrução de objetos em 3D agora pode ser complementado com etapas de replicação de objetos para o mundo real. Esta área de estudo encontra-se em expansão, sobretudo no contexto da preservação do patrimônio cultural. No contexto de museus e entidades de preservação de acervos culturais, é particularmente interessante que se possa capturar as coleções em 3D.
A captura em 3D possibilita que o pesquisador possa manipular a peça virtual como se fosse a original. A partir daí, pode realizar estudos detalhados sobre como a peça foi construída, entendendo melhor as suas características. Para o usuário comum, a manipulação virtual de uma peça pode tazer uma experiência muito próxima da real, sem o risco de danificar. Há ainda a possibilidade de se replicar uma peça existente, através da impressão 3D, possibilitando trazer para o mundo real objetos que foram capturados remotamente por um museu em local geograficamente distante, além de possibilitar a experiência sensorial completa de manipular uma peça rara, sem o risco de danificá-la.
Ampliando ainda mais o realismo, a hidropintura pode conferir um acabamento mais realista possível a uma peça, uma vez que peças produzidas por uma impressora em 3D possui uma cor única.

Impressõesm em 3D com cores pode existir, porém seu custo é elevado comparado a produção de peças com hidropintura. Também,


Em computação gráfica, através de um conjunto de técnicas de mapeamento de padronagens de cores a um objeto modelado em 3D, é possível criar imagens com cores realistas a um objeto real.

Nos dias atuais, é possível criar, modelar, reproduzir e protitipar objetos em 3D através da impressão 3D juntamente com as técnicas de reconstrução em 3D.
Com a tecnologia atual de baixo custo, existem limitações a respeito das cor do objeto criado através da impressão 3D.

O processo de hidropintura inicia-se com o filme já previamente impresso colocado em um recipiente com água morna. Neste filme, é aplicado um ativador que o disolve, tornando-o uma película fina e adesiva na superfície da água.




Trabalho do  \ac{PGCOMP}. Bolsa do \ac{CNPq}.

\begin{figure}[h]
    Figure
    \caption{As siglas também funcionam nas legendas, seja na forma de sigla \ac{CNPq}, seja na forma completa \acf{PGCOMP}.}
\end{figure}


\subsection{Uma Subse\c{c}\~{a}o}
\acresetall
Texto para mostrar como o \verb|\acresetall| funciona \ac{CNPq}, \ac{PGCOMP}. Ele reseta os contadoes e faz a sigla aparecer na forma estendida novamente.

\subsection{Outra Subse\c{c}\~{a}o}

Texto  \acf{CNPq}, \acf{PGCOMP}.
