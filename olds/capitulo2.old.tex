\xchapter{Trabalhos Relacionados}{Neste capítulo, serão mencionados os trabalhos relacionados ao objeto de pesquisa, suas contribuições, algumas considerações, possíveis assuntos que serão incorporados na proposta de pesquisa}

\acresetall 

\section{Simulação da Física}

O processo da impressão hidrográfica é um fenômeno físico que envolve a interação entre dois objetos principais: um deles é a superfície que está sendo pintada, o outro é o filme que carrega o pigmento. O filme permanece flutuando na superfície da água, formando um composto imiscível. Comporta-se como uma lâmina viscosa que ao tocar na superfície do objeto, adere quase que imediatamente à superfície deste objeto, acompanhando o movimento de imersão.

Foram selecionados alguns trabalhos relevantes para estudo de como conceber uma simulação baseada em física que represente este tipo de interação entre corpos.

Em \cite{Batty2012}, é apresentado um modelo discreto que representa a dinâmica de fluidos que se apresentem como uma lâmina viscosa, com a preservação do volume. Neste trabalho, a lâmina viscosa é apresentada como um corpo líquido sem forma delimitada, com volume constante, com um grau de elasticidade, uma tensão de superfície e que pode eventualmente interagir com outros corpos rígidos. A proposta modela as lâminas viscosas como malhas de triângulos, que representam líquido em questão. Com este modelo é possível executar uma simulação discretizada, porém ainda consistente com a física, das forças envolvidas neste tipo de sistema. Uma variedade de líquidos que dobram ou que se enrolam quando derramados (mel, glace, lava) podem ser simulados com base neste trabalho.

No trabalho \cite{muller2007position} é apresentada a técnica de \ac{PBD}, uma tratamento diferente para se modelar a simulação física. O sistema de forças (tradicional na simulação com base das leis de Newton) fica em segundo plano nesta abordagem. É feita a modelagem de um conjunto de posições e restrições que atuam nestas posições. A atualização da cena é feita modificando-se diretamente as posições dos corpos envolvidos. O algoritmo proposto guarda muitas similaridades com o método de Euler, contudo evita a instabilidade gerada no sistema por conta do uso das equações do movimento de Newton, que necessitam de derivações até a segunda ordem para finalmente determinar as posições dos corpos da cena. Ainda oferece uma alternativa menos custosa que o \ac{MEF} para simular corpos deformáveis, corpos rígidos e fluidos, dentre outros sistemas considerados complexos.

Esse trabalho é complementado posteriormente na proposta explicada em \cite{fratarcangeli2013gpu}, que oferece uma solução para paralelizar o algoritmo de \ac{PBD}. A proposta leva a uma redução do uso da \ac{CPU} e a distribuição do processamento para os vários núcleos disponíveis na \ac{GPU}. Isto deixa a execução da animação sem interrupções e abre a possibilidade para a implementação de \ac{PBD} para simulações em tempo real.

Outra complementação importante para a técnica \ac{PBD} foi feita mais tarde por\cite{macklin2016xpbd}, que propôs um método simples para tornar a rigidez dos corpos independente do intervalo de tempo determinado na simulação e independente da quantidade de iterações determinada para resolução das \textit{contraints}. Com isto, o aspecto visual de elasticidade dos corpos em uma cena torna-se mais controlável através de um parâmetro independente, específico para este fim.



\section{Mapeamento de Texturas}

Mapeamento de texturas é um procedimento bastante utilizado na computação gráfica \cite{haeberli1993texture}. O principal objetivo desta técnica é adicionar realismo a uma cena. A forma mais simples de mapeamento de textura consiste em aplicar uma imagem 2D a um objeto 3D da cena. \cite[Capítulo~6]{shreiner2013opengl}

Para realização desta técnica, é determinado o sistema de coordenadas da imagem 2D (a textura) e o sistema de coordenadas do objeto 3D. Em seguida, define-se a função que fará o mapeamento das coordenadas da textura para o objeto. Existem diversas maneiras de se definir a função de mapeamento conforme mostrado em \cite{hormann2007mesh}.

%\section{Impressão 3D}
%A impressão 3D vem se desenvolvendo nos últimos anos.  Existem trabalhos bastante avançados denotando um grande domínio da fabricação aditiva e \textit{design} de produtos com as %características desejadas em termo de formas e estruturas \cite{stava2012stress}.

%\section{Considerações}
%Os trabalhos realizados na área de simulação trouxeram contribuições relevantes que podem ser aproveitadas na proposição de  das técnicas de pintura em 3D.
%Os trabalhos também deixam em aberto problemas encontrados como:
%O filme torna-se difícil de controlar quando o ângulo de imersão é superior a 45\degree
%Sugrir ao usuário uma posição para a imersão
%Dificuldade no posiconameto do filme
%Calibragem da cor
%Reprodutibilidade
%Tratamento de concavidades
%Quantificação do esticamento do filme em uma condição de variação não linear
