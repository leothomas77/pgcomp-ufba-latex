\xchapter{Defini\c{c}\~{a}o da proposta}{} %sem preambulo

Texto da introdução com a contextualização, o marco teórico e a formulação de hipóteses

%A numera\c{c}\~{a}o de figuras \'{e} sequencial, dentro do cap\'{\i}tulo. Ver Figura \ref{default-regular1} e Figura \ref{default-regular2}.

%A numera\c{c}\~{a}o de tabelaas \'{e} sequencial, dentro do cap\'{\i}tulo. Ver Tabela \ref{default-table1} e Tabela \ref{default-table2}.


\section{Exemplos de Figura}

\begin{figure}[htbp]
\begin{center}
  \includegraphics[scale=0.5]{ufba.eps}
\caption{Bras\~{a}o da UFBA - Menor.}
\label{default-regular1}
\end{center}
\end{figure}

\begin{figure}[htbp]
\begin{center}
  \includegraphics[scale=0.75]{ufba.eps}
\caption{Bras\~{a}o da UFBA - Maior.}
\label{default-regular2}
\end{center}
\end{figure}

\lipsum

\section{Exemplos de Tabela}
\subsection{Uma Tabela}
\begin{table}[htbp]
\caption{Uma tabela com 3 linhas e 2 colunas.}
\begin{center}
\begin{tabular}{|c|c|} 
\hline
elemento 11 & elemento 12 \\ \hline
elemento 21 & elemento 22 \\ \hline
elemento 31 & elemento 32 \\
\hline
\end{tabular}
\end{center}
\label{default-table1}
\end{table}

%\lipsum

\begin{table}[htbp]
\caption{Uma tabela com 3 linhas e 3 colunas.}
\begin{center}
\begin{tabular}{|l|c|c|} 
\hline
elemento 11 & elemento 12 & elemento 13\\ \hline
elemento 21 & elemento 22 & elemento 23\\ \hline
elemento 31 & elemento 32 & elemento 33\\
\hline
\end{tabular}
\end{center}
\label{default-table2}
\end{table}%