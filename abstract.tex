\abstract

Hydrographic printing, Water Transfer Printing or Hydrographics is a viable technique for coloring objects created with 3D printers. However, in this type of painting there is a complex interaction between a film and a 3D printed object, in wich the image contained in the film must adhere over the object. This process is subjected to several uncertainties about how the film will deform to adhere over the 3D object and transfer the image to the surface of the object, giving it a finish resulting from a mapping of a 2D image into a real 3D model.

Faced with these difficulties, a 3D computational simulation proposal will be presented for the hydrographic printing technique. Based on the physical interaction between the bodies involved in this phenomenon, it will be proposed a 3D computer simulation that shows what happens during hydrographic printing. In this way, the user will be given a support to preview the painting process in more complex shapes, which require alignments with the image transferred to the object. Futhermore, the work also aims to simulate the physical phenomenon of hydrographic printing in real time, making an evaluation of the use of the computational resources of the \ac{GPU} used in this simulation.

% Palavras-chave do resumo em Ingles
\begin{keywords}
3D printing, hydrographic printing, physically-based animation, real-time rendering, texture mapping
\end{keywords}
